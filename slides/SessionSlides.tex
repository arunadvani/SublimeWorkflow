\documentclass{beamer}
\definecolor{mid-gray}{gray}{0.6}
\usetheme{WarwickCAGEIFSSlides}
\usepackage{Arun_general_slides}
\AtBeginSection{\frame{\sectionpage}}

%Define all the title things
\title[Sublime Session]{Sublime Session}
\author[Advani]{Arun~Advani}
\institute[Warwick, CAGE \& IFS]{University of Warwick, CAGE Research Centre and Institute for Fiscal Studies}
\date{12/05/2020}



%%%%%%%%%%%%%%%%%%%
%%%% BEGIN DOC %%%%
%%%%%%%%%%%%%%%%%%%
\begin{document}

%Declare a titlepage slide
\begin{frame}
\titlepage
\end{frame}


%%%%%%%%%%%%%
% INTRO
%%%%%%%%%%%%%

% USUAL INTRO STRUCTURE
% \frametitle{Motivation}
% \frametitle{Questions}
% \frametitle{Preview}
% \frametitle{Contribution and Related Lit}

\begin{frame}
\frametitle{Why sublime} \label{Why sublime}
	\begin{enumerate}
		\item Learn one set of shortcuts, rule the world.
		\item Edit lots of stuff simultaneously.
		\item Think of a problem, find a package.
	\end{enumerate}
\end{frame}


\begin{frame}
\frametitle{What can sublime do?} \label{What can sublime do?}
	\begin{itemize}
		\item write: plaintext, markdown, LaTeX/Beamer
		\item code: stata, R, Julia, html/css 
		\item edit: easy to do repetitive tasks
		\item notes and todo lists
		\item ... above is non-exhaustive, but covers lots of my use cases
	\end{itemize}
\end{frame}


% \begin{frame}
% \frametitle{Session Plan} \label{Session Plan}
% 	Best way to see the value of sublime is to use it.
% 	\begin{enumerate}
% 		\item create a text file
% 		\item create a LaTeX article
% 		\item 
% 	\end{enumerate}
% \end{frame}

\begin{frame}
\frametitle{Getting started} \label{Getting started}
	\begin{itemize}
		\item Let's clone the repo: https://github.com/arunadvani/SublimeWorkflow.git
		\begin{itemize}
		 	\item Means we can have some fun with git cmd
		 	\item NOTE, if you edit the path, you can call sublime from here. 
		 \end{itemize} 
	\end{itemize}
\end{frame}

\begin{frame}
\frametitle{Steps in cloning} \label{Steps in cloning}
	\begin{itemize}
		\item \texttt{mkdir}
		\item \texttt{cd <dir>}
		\item \texttt{git clone https://github.com/arunadvani/SublimeWorkflow.git}
		\item \texttt{explorer .}
		\item \texttt{subl .} [need to edit path first]
	\end{itemize}
\end{frame}

\begin{frame}
\frametitle{Create a README} \label{Create a README}
	\begin{itemize}
		\item I foolishly created a repo without a readme!
		\item Let's fix that 
		\begin{itemize}
			\item can create a readme (markdown)
			\item and then push it (though for big commits I usu go via git cmd)
			\item note: let's not all try to push to repo simultaneously!
		\end{itemize}
		\item markdown is plaintext, so don't need to install a package. but \texttt{MarkdownEditing} makes it look nicer.
	\end{itemize}
\end{frame}

\begin{frame}
\frametitle{Make a quick LaTeX article} \label{Make a quick LaTeX article}
	Need \texttt{LaTeXTools} for this one. \texttt{LaTeX Word Count} and \texttt{Text Pastry} also helpful
	\begin{itemize}
		\item we'll make a quick list + add many numbers
		\item and then reorder it
		\item and sort out some indentation + plus multiedits + copying
		\item comment out the bibliography before you build
		\item (can look at some jumping by editing the slides)
	\end{itemize}
\end{frame}

\begin{frame}
\frametitle{Stata} \label{Stata}
	\begin{itemize}
		\item my very favourite thing.
		\item You should all already have \texttt{StataEditor}. (can be a pain to set up)
		\item let's say ``\texttt{Hello, world!}''
		\item from Stata 17 we can integrate into sublime (and/or notebooks!)
	\end{itemize}
\end{frame}

\begin{frame}
\frametitle{R} \label{R}
	\begin{itemize}
		\item useful to create a separate window for the REPL
		\begin{itemize}
			\item we'll need \texttt{Origami} and \texttt{Terminus} for that
		\end{itemize}
		\item now can write and run some code 
		\begin{itemize}
			\item need \texttt{R-IDE}
			\item let's say ``\texttt{Hello, world!}'' again
		\end{itemize}
	\end{itemize}
\end{frame}

\begin{frame}
\frametitle{Editing} \label{Editing}
	\begin{itemize}
		\item Let's sort out a LaTeX table
		\begin{itemize}
			\item I hate these
			\item But sublime makes it much easier
			\item \texttt{AlignTab} is our friend
		\end{itemize}
	\end{itemize}
\end{frame}

\begin{frame}
\frametitle{Notes and ToDos} \label{Notes and ToDos}
	\begin{itemize}
		\item \texttt{PlainNotes} for notes
		\item \texttt{PlainTasks} for to do list
	\end{itemize}
\end{frame}

\end{document}